%% bare_jrnl.tex
%% V1.4b
%% 2015/08/26
%% by Michael Shell
%% see http://www.michaelshell.org/
%% for current contact information.
%%
%% This is a skeleton file demonstrating the use of IEEEtran.cls
%% (requires IEEEtran.cls version 1.8b or later) with an IEEE
%% journal paper.
%%
%% Support sites:
%% http://www.michaelshell.org/tex/ieeetran/
%% http://www.ctan.org/pkg/ieeetran
%% and
%% http://www.ieee.org/

%%*************************************************************************
%% Legal Notice:
%% This code is offered as-is without any warranty either expressed or
%% implied; without even the implied warranty of MERCHANTABILITY or
%% FITNESS FOR A PARTICULAR PURPOSE! 
%% User assumes all risk.
%% In no event shall the IEEE or any contributor to this code be liable for
%% any damages or losses, including, but not limited to, incidental,
%% consequential, or any other damages, resulting from the use or misuse
%% of any information contained here.
%%
%% All comments are the opinions of their respective authors and are not
%% necessarily endorsed by the IEEE.
%%
%% This work is distributed under the LaTeX Project Public License (LPPL)
%% ( http://www.latex-project.org/ ) version 1.3, and may be freely used,
%% distributed and modified. A copy of the LPPL, version 1.3, is included
%% in the base LaTeX documentation of all distributions of LaTeX released
%% 2003/12/01 or later.
%% Retain all contribution notices and credits.
%% ** Modified files should be clearly indicated as such, including  **
%% ** renaming them and changing author support contact information. **
%%*************************************************************************


% *** Authors should verify (and, if needed, correct) their LaTeX system  ***
% *** with the testflow diagnostic prior to trusting their LaTeX platform ***
% *** with production work. The IEEE's font choices and paper sizes can   ***
% *** trigger bugs that do not appear when using other class files.       ***                          ***
% The testflow support page is at:
% http://www.michaelshell.org/tex/testflow/



\documentclass[journal]{IEEEtran}
%
% If IEEEtran.cls has not been installed into the LaTeX system files,
% manually specify the path to it like:
% \documentclass[journal]{../sty/IEEEtran}





% Some very useful LaTeX packages include:
% (uncomment the ones you want to load)


% *** MISC UTILITY PACKAGES ***
%
%\usepackage{ifpdf}
% Heiko Oberdiek's ifpdf.sty is very useful if you need conditional
% compilation based on whether the output is pdf or dvi.
% usage:
% \ifpdf
%   % pdf code
% \else
%   % dvi code
% \fi
% The latest version of ifpdf.sty can be obtained from:
% http://www.ctan.org/pkg/ifpdf
% Also, note that IEEEtran.cls V1.7 and later provides a builtin
% \ifCLASSINFOpdf conditional that works the same way.
% When switching from latex to pdflatex and vice-versa, the compiler may
% have to be run twice to clear warning/error messages.






% *** CITATION PACKAGES ***
%
%\usepackage{cite}
% cite.sty was written by Donald Arseneau
% V1.6 and later of IEEEtran pre-defines the format of the cite.sty package
% \cite{} output to follow that of the IEEE. Loading the cite package will
% result in citation numbers being automatically sorted and properly
% "compressed/ranged". e.g., [1], [9], [2], [7], [5], [6] without using
% cite.sty will become [1], [2], [5]--[7], [9] using cite.sty. cite.sty's
% \cite will automatically add leading space, if needed. Use cite.sty's
% noadjust option (cite.sty V3.8 and later) if you want to turn this off
% such as if a citation ever needs to be enclosed in parenthesis.
% cite.sty is already installed on most LaTeX systems. Be sure and use
% version 5.0 (2009-03-20) and later if using hyperref.sty.
% The latest version can be obtained at:
% http://www.ctan.org/pkg/cite
% The documentation is contained in the cite.sty file itself.






% *** GRAPHICS RELATED PACKAGES ***
%
\ifCLASSINFOpdf
  % \usepackage[pdftex]{graphicx}
  % declare the path(s) where your graphic files are
  % \graphicspath{{../pdf/}{../jpeg/}}
  % and their extensions so you won't have to specify these with
  % every instance of \includegraphics
  % \DeclareGraphicsExtensions{.pdf,.jpeg,.png}
\else
  % or other class option (dvipsone, dvipdf, if not using dvips). graphicx
  % will default to the driver specified in the system graphics.cfg if no
  % driver is specified.
  % \usepackage[dvips]{graphicx}
  % declare the path(s) where your graphic files are
  % \graphicspath{{../eps/}}
  % and their extensions so you won't have to specify these with
  % every instance of \includegraphics
  % \DeclareGraphicsExtensions{.eps}
\fi
% graphicx was written by David Carlisle and Sebastian Rahtz. It is
% required if you want graphics, photos, etc. graphicx.sty is already
% installed on most LaTeX systems. The latest version and documentation
% can be obtained at: 
% http://www.ctan.org/pkg/graphicx
% Another good source of documentation is "Using Imported Graphics in
% LaTeX2e" by Keith Reckdahl which can be found at:
% http://www.ctan.org/pkg/epslatex
%
% latex, and pdflatex in dvi mode, support graphics in encapsulated
% postscript (.eps) format. pdflatex in pdf mode supports graphics
% in .pdf, .jpeg, .png and .mps (metapost) formats. Users should ensure
% that all non-photo figures use a vector format (.eps, .pdf, .mps) and
% not a bitmapped formats (.jpeg, .png). The IEEE frowns on bitmapped formats
% which can result in "jaggedy"/blurry rendering of lines and letters as
% well as large increases in file sizes.
%
% You can find documentation about the pdfTeX application at:
% http://www.tug.org/applications/pdftex





% *** MATH PACKAGES ***
%
%\usepackage{amsmath}
% A popular package from the American Mathematical Society that provides
% many useful and powerful commands for dealing with mathematics.
%
% Note that the amsmath package sets \interdisplaylinepenalty to 10000
% thus preventing page breaks from occurring within multiline equations. Use:
%\interdisplaylinepenalty=2500
% after loading amsmath to restore such page breaks as IEEEtran.cls normally
% does. amsmath.sty is already installed on most LaTeX systems. The latest
% version and documentation can be obtained at:
% http://www.ctan.org/pkg/amsmath





% *** SPECIALIZED LIST PACKAGES ***
%
%\usepackage{algorithmic}
% algorithmic.sty was written by Peter Williams and Rogerio Brito.
% This package provides an algorithmic environment fo describing algorithms.
% You can use the algorithmic environment in-text or within a figure
% environment to provide for a floating algorithm. Do NOT use the algorithm
% floating environment provided by algorithm.sty (by the same authors) or
% algorithm2e.sty (by Christophe Fiorio) as the IEEE does not use dedicated
% algorithm float types and packages that provide these will not provide
% correct IEEE style captions. The latest version and documentation of
% algorithmic.sty can be obtained at:
% http://www.ctan.org/pkg/algorithms
% Also of interest may be the (relatively newer and more customizable)
% algorithmicx.sty package by Szasz Janos:
% http://www.ctan.org/pkg/algorithmicx




% *** ALIGNMENT PACKAGES ***
%
%\usepackage{array}
% Frank Mittelbach's and David Carlisle's array.sty patches and improves
% the standard LaTeX2e array and tabular environments to provide better
% appearance and additional user controls. As the default LaTeX2e table
% generation code is lacking to the point of almost being broken with
% respect to the quality of the end results, all users are strongly
% advised to use an enhanced (at the very least that provided by array.sty)
% set of table tools. array.sty is already installed on most systems. The
% latest version and documentation can be obtained at:
% http://www.ctan.org/pkg/array


% IEEEtran contains the IEEEeqnarray family of commands that can be used to
% generate multiline equations as well as matrices, tables, etc., of high
% quality.




% *** SUBFIGURE PACKAGES ***
%\ifCLASSOPTIONcompsoc
%  \usepackage[caption=false,font=normalsize,labelfont=sf,textfont=sf]{subfig}
%\else
%  \usepackage[caption=false,font=footnotesize]{subfig}
%\fi
% subfig.sty, written by Steven Douglas Cochran, is the modern replacement
% for subfigure.sty, the latter of which is no longer maintained and is
% incompatible with some LaTeX packages including fixltx2e. However,
% subfig.sty requires and automatically loads Axel Sommerfeldt's caption.sty
% which will override IEEEtran.cls' handling of captions and this will result
% in non-IEEE style figure/table captions. To prevent this problem, be sure
% and invoke subfig.sty's "caption=false" package option (available since
% subfig.sty version 1.3, 2005/06/28) as this is will preserve IEEEtran.cls
% handling of captions.
% Note that the Computer Society format requires a larger sans serif font
% than the serif footnote size font used in traditional IEEE formatting
% and thus the need to invoke different subfig.sty package options depending
% on whether compsoc mode has been enabled.
%
% The latest version and documentation of subfig.sty can be obtained at:
% http://www.ctan.org/pkg/subfig




% *** FLOAT PACKAGES ***
%
%\usepackage{fixltx2e}
% fixltx2e, the successor to the earlier fix2col.sty, was written by
% Frank Mittelbach and David Carlisle. This package corrects a few problems
% in the LaTeX2e kernel, the most notable of which is that in current
% LaTeX2e releases, the ordering of single and double column floats is not
% guaranteed to be preserved. Thus, an unpatched LaTeX2e can allow a
% single column figure to be placed prior to an earlier double column
% figure.
% Be aware that LaTeX2e kernels dated 2015 and later have fixltx2e.sty's
% corrections already built into the system in which case a warning will
% be issued if an attempt is made to load fixltx2e.sty as it is no longer
% needed.
% The latest version and documentation can be found at:
% http://www.ctan.org/pkg/fixltx2e


%\usepackage{stfloats}
% stfloats.sty was written by Sigitas Tolusis. This package gives LaTeX2e
% the ability to do double column floats at the bottom of the page as well
% as the top. (e.g., "\begin{figure*}[!b]" is not normally possible in
% LaTeX2e). It also provides a command:
%\fnbelowfloat
% to enable the placement of footnotes below bottom floats (the standard
% LaTeX2e kernel puts them above bottom floats). This is an invasive package
% which rewrites many portions of the LaTeX2e float routines. It may not work
% with other packages that modify the LaTeX2e float routines. The latest
% version and documentation can be obtained at:
% http://www.ctan.org/pkg/stfloats
% Do not use the stfloats baselinefloat ability as the IEEE does not allow
% \baselineskip to stretch. Authors submitting work to the IEEE should note
% that the IEEE rarely uses double column equations and that authors should try
% to avoid such use. Do not be tempted to use the cuted.sty or midfloat.sty
% packages (also by Sigitas Tolusis) as the IEEE does not format its papers in
% such ways.
% Do not attempt to use stfloats with fixltx2e as they are incompatible.
% Instead, use Morten Hogholm'a dblfloatfix which combines the features
% of both fixltx2e and stfloats:
%
% \usepackage{dblfloatfix}
% The latest version can be found at:
% http://www.ctan.org/pkg/dblfloatfix




%\ifCLASSOPTIONcaptionsoff
%  \usepackage[nomarkers]{endfloat}
% \let\MYoriglatexcaption\caption
% \renewcommand{\caption}[2][\relax]{\MYoriglatexcaption[#2]{#2}}
%\fi
% endfloat.sty was written by James Darrell McCauley, Jeff Goldberg and 
% Axel Sommerfeldt. This package may be useful when used in conjunction with 
% IEEEtran.cls'  captionsoff option. Some IEEE journals/societies require that
% submissions have lists of figures/tables at the end of the paper and that
% figures/tables without any captions are placed on a page by themselves at
% the end of the document. If needed, the draftcls IEEEtran class option or
% \CLASSINPUTbaselinestretch interface can be used to increase the line
% spacing as well. Be sure and use the nomarkers option of endfloat to
% prevent endfloat from "marking" where the figures would have been placed
% in the text. The two hack lines of code above are a slight modification of
% that suggested by in the endfloat docs (section 8.4.1) to ensure that
% the full captions always appear in the list of figures/tables - even if
% the user used the short optional argument of \caption[]{}.
% IEEE papers do not typically make use of \caption[]'s optional argument,
% so this should not be an issue. A similar trick can be used to disable
% captions of packages such as subfig.sty that lack options to turn off
% the subcaptions:
% For subfig.sty:
% \let\MYorigsubfloat\subfloat
% \renewcommand{\subfloat}[2][\relax]{\MYorigsubfloat[]{#2}}
% However, the above trick will not work if both optional arguments of
% the \subfloat command are used. Furthermore, there needs to be a
% description of each subfigure *somewhere* and endfloat does not add
% subfigure captions to its list of figures. Thus, the best approach is to
% avoid the use of subfigure captions (many IEEE journals avoid them anyway)
% and instead reference/explain all the subfigures within the main caption.
% The latest version of endfloat.sty and its documentation can obtained at:
% http://www.ctan.org/pkg/endfloat
%
% The IEEEtran \ifCLASSOPTIONcaptionsoff conditional can also be used
% later in the document, say, to conditionally put the References on a 
% page by themselves.




% *** PDF, URL AND HYPERLINK PACKAGES ***
%
%\usepackage{url}
% url.sty was written by Donald Arseneau. It provides better support for
% handling and breaking URLs. url.sty is already installed on most LaTeX
% systems. The latest version and documentation can be obtained at:
% http://www.ctan.org/pkg/url
% Basically, \url{my_url_here}.




% *** Do not adjust lengths that control margins, column widths, etc. ***
% *** Do not use packages that alter fonts (such as pslatex).         ***
% There should be no need to do such things with IEEEtran.cls V1.6 and later.
% (Unless specifically asked to do so by the journal or conference you plan
% to submit to, of course. )


% correct bad hyphenation here
\hyphenation{op-tical net-works semi-conduc-tor}


\begin{document}
%
% paper title
% Titles are generally capitalized except for words such as a, an, and, as,
% at, but, by, for, in, nor, of, on, or, the, to and up, which are usually
% not capitalized unless they are the first or last word of the title.
% Linebreaks \\ can be used within to get better formatting as desired.
% Do not put math or special symbols in the title.
\title{Bare Demo of IEEEtran.cls\\ for IEEE Journals}
%
%
% author names and IEEE memberships
% note positions of commas and nonbreaking spaces ( ~ ) LaTeX will not break
% a structure at a ~ so this keeps an author's name from being broken across
% two lines.
% use \thanks{} to gain access to the first footnote area
% a separate \thanks must be used for each paragraph as LaTeX2e's \thanks
% was not built to handle multiple paragraphs
%

\author{Ana~Camarinha,
        Daniel~Proaño-Guevara,~\IEEEmembership{Member,~IEEE,}
        Evangelia~Antoniadi,
        Miguel~Trindade~Campos% <-this % stops a space

\thanks{All authors contributed equally for this work. Their names appear in alphabetical order.}% <-this % stops a space
\thanks{Manuscript submitted on December 1, 2025.}}

% note the % following the last \IEEEmembership and also \thanks - 
% these prevent an unwanted space from occurring between the last author name
% and the end of the author line. i.e., if you had this:
% 
% \author{....lastname \thanks{...} \thanks{...} }
%                     ^------------^------------^----Do not want these spaces!
%
% a space would be appended to the last name and could cause every name on that
% line to be shifted left slightly. This is one of those "LaTeX things". For
% instance, "\textbf{A} \textbf{B}" will typeset as "A B" not "AB". To get
% "AB" then you have to do: "\textbf{A}\textbf{B}"
% \thanks is no different in this regard, so shield the last } of each \thanks
% that ends a line with a % and do not let a space in before the next \thanks.
% Spaces after \IEEEmembership other than the last one are OK (and needed) as
% you are supposed to have spaces between the names. For what it is worth,
% this is a minor point as most people would not even notice if the said evil
% space somehow managed to creep in.



% The paper headers
\markboth{Report for Seminars in Biomedical Engineering 2024-2025, PRODEB, FEUP, Group 1}%
{Group_1: Report for SEB 2024/25}
% The only time the second header will appear is for the odd numbered pages
% after the title page when using the twoside option.
% 
% *** Note that you probably will NOT want to include the author's ***
% *** name in the headers of peer review papers.                   ***
% You can use \ifCLASSOPTIONpeerreview for conditional compilation here if
% you desire.




% If you want to put a publisher's ID mark on the page you can do it like
% this:
%\IEEEpubid{0000--0000/00\$00.00~\copyright~2015 IEEE}
% Remember, if you use this you must call \IEEEpubidadjcol in the second
% column for its text to clear the IEEEpubid mark.



% use for special paper notices
%\IEEEspecialpapernotice{(Invited Paper)}




% make the title area
\maketitle

% As a general rule, do not put math, special symbols or citations
% in the abstract or keywords.
\begin{abstract}
The abstract goes here.
\end{abstract}

% Note that keywords are not normally used for peerreview papers.
\begin{IEEEkeywords}
IEEE, IEEEtran, journal, \LaTeX, paper, template.
\end{IEEEkeywords}






% For peer review papers, you can put extra information on the cover
% page as needed:
% \ifCLASSOPTIONpeerreview
% \begin{center} \bfseries EDICS Category: 3-BBND \end{center}
% \fi
%
% For peerreview papers, this IEEEtran command inserts a page break and
% creates the second title. It will be ignored for other modes.
% \IEEEpeerreviewmaketitle



\section{Introduction}
% The very first letter is a 2 line initial drop letter followed
% by the rest of the first word in caps.
% 
% form to use if the first word consists of a single letter:
% \IEEEPARstart{A}{demo} file is ....
% 
% form to use if you need the single drop letter followed by
% normal text (unknown if ever used by the IEEE):
% \IEEEPARstart{A}{}demo file is ....
% 
% Some journals put the first two words in caps:
% \IEEEPARstart{T}{his demo} file is ....
% 
% Here we have the typical use of a "T" for an initial drop letter
% and "HIS" in caps to complete the first word.
\IEEEPARstart{T}{his} demo file is intended to serve as a ``starter file''
for IEEE journal papers produced under \LaTeX\ using
IEEEtran.cls version 1.8b and later.
% You must have at least 2 lines in the paragraph with the drop letter
% (should never be an issue)
I wish you the best of success.


\hfill mds
 
\hfill August 26, 2015

\subsection{Subsection Heading Here}
Subsection text here.

% needed in second column of first page if using \IEEEpubid
%\IEEEpubidadjcol

\subsubsection{Subsubsection Heading Here}
Subsubsection text here.


- Overview of Wearable Medical Devices

- Significance in healthcare and daily monitoring

- Objectives of the review

- Methodology of how literature was gathered


% An example of a floating figure using the graphicx package.
% Note that \label must occur AFTER (or within) \caption.
% For figures, \caption should occur after the \includegraphics.
% Note that IEEEtran v1.7 and later has special internal code that
% is designed to preserve the operation of \label within \caption
% even when the captionsoff option is in effect. However, because
% of issues like this, it may be the safest practice to put all your
% \label just after \caption rather than within \caption{}.
%
% Reminder: the "draftcls" or "draftclsnofoot", not "draft", class
% option should be used if it is desired that the figures are to be
% displayed while in draft mode.
%
%\begin{figure}[!t]
%\centering
%\includegraphics[width=2.5in]{myfigure}
% where an .eps filename suffix will be assumed under latex, 
% and a .pdf suffix will be assumed for pdflatex; or what has been declared
% via \DeclareGraphicsExtensions.
%\caption{Simulation results for the network.}
%\label{fig_sim}
%\end{figure}

% Note that the IEEE typically puts floats only at the top, even when this
% results in a large percentage of a column being occupied by floats.


% An example of a double column floating figure using two subfigures.
% (The subfig.sty package must be loaded for this to work.)
% The subfigure \label commands are set within each subfloat command,
% and the \label for the overall figure must come after \caption.
% \hfil is used as a separator to get equal spacing.
% Watch out that the combined width of all the subfigures on a 
% line do not exceed the text width or a line break will occur.
%
%\begin{figure*}[!t]
%\centering
%\subfloat[Case I]{\includegraphics[width=2.5in]{box}%
%\label{fig_first_case}}
%\hfil
%\subfloat[Case II]{\includegraphics[width=2.5in]{box}%
%\label{fig_second_case}}
%\caption{Simulation results for the network.}
%\label{fig_sim}
%\end{figure*}
%
% Note that often IEEE papers with subfigures do not employ subfigure
% captions (using the optional argument to \subfloat[]), but instead will
% reference/describe all of them (a), (b), etc., within the main caption.
% Be aware that for subfig.sty to generate the (a), (b), etc., subfigure
% labels, the optional argument to \subfloat must be present. If a
% subcaption is not desired, just leave its contents blank,
% e.g., \subfloat[].


% An example of a floating table. Note that, for IEEE style tables, the
% \caption command should come BEFORE the table and, given that table
% captions serve much like titles, are usually capitalized except for words
% such as a, an, and, as, at, but, by, for, in, nor, of, on, or, the, to
% and up, which are usually not capitalized unless they are the first or
% last word of the caption. Table text will default to \footnotesize as
% the IEEE normally uses this smaller font for tables.
% The \label must come after \caption as always.
%
%\begin{table}[!t]
%% increase table row spacing, adjust to taste
%\renewcommand{\arraystretch}{1.3}
% if using array.sty, it might be a good idea to tweak the value of
% \extrarowheight as needed to properly center the text within the cells
%\caption{An Example of a Table}
%\label{table_example}
%\centering
%% Some packages, such as MDW tools, offer better commands for making tables
%% than the plain LaTeX2e tabular which is used here.
%\begin{tabular}{|c||c|}
%\hline
%One & Two\\
%\hline
%Three & Four\\
%\hline
%\end{tabular}
%\end{table}


% Note that the IEEE does not put floats in the very first column
% - or typically anywhere on the first page for that matter. Also,
% in-text middle ("here") positioning is typically not used, but it
% is allowed and encouraged for Computer Society conferences (but
% not Computer Society journals). Most IEEE journals/conferences use
% top floats exclusively. 
% Note that, LaTeX2e, unlike IEEE journals/conferences, places
% footnotes above bottom floats. This can be corrected via the
% \fnbelowfloat command of the stfloats package.

\section{What are Wearable Medical Devices?}
    \subsection{Definition and Purpose}

    Wearable Medical Devices (WMDs) are advanced tools designed to provide continuous monitoring of various physiological parameters without disrupting users' daily routines. These devices integrate seamlessly into daily life, allowing monitoring of vital signs during activities such as work or exercise, and are also applicable in clinical settings~\cite{Fotiadis2006}. The development of WMDs has been driven by rapid advances in biomedical technologies, micro and nanotechnologies, materials engineering, electronic systems, and information technology, resulting in increased comfort, precision, and widespread adoption of these devices~\cite{Degerli2020, Fotiadis2006}. In 2022, more than a billion wearable medical devices were in use worldwide.

    The term "wearable" encompasses devices that are worn directly on the body or integrated into clothing, while "medical device" refers to the tools used for medical functions such as monitoring, aiding recovery, or supporting long-term care. WMDs are designed to be autonomous, non-invasive and tailored to support these medical functions, ultimately aiming to improve patient health~\cite{Degerli2020}. According to the Food and Drug Administration (FDA), a medical device must perform its intended function without relying on drugs or other biological substances, positioning WMDs as highly diverse - from simple wearable sensors to sophisticated electrodes for cardiac monitoring~\cite{Khan2016, Ates2022}.

    \subsection{Categories of Wearable Devices}
    
    Wearable medical devices can be classified into three main categories according to their primary purpose: monitoring devices, medical aids, and rehabilitation devices~\cite{Fotiadis2006}.
    
        \subsubsection{Wearable Monitoring Devices}
        These devices are used to monitor and manage chronic diseases and measure vital signs such as heart rate, oxygen saturation, respiration rate, and body fat. They provide critical data that help healthcare professionals make informed decisions about patient care. Examples include smart watches with health tracking capabilities and portable ECG monitors.

        \subsubsection{Wearable Medical Aids}
        Designed for patients with disabilities, these devices provide ongoing support to people with temporary or permanent physical limitations. Examples include hearing aids and contact lenses, which help improve daily functioning and improve patient quality of life.

        \subsubsection{Wearable Rehabilitation Devices}
        These devices are often used in patients recovering after surgery or other high-risk situations. Rehabilitation wearables combine monitoring features with assistive functions to support the patient during recovery. Examples include exoskeletons and other devices that help regain mobility and strengthen muscles~\cite{Hemapriya2017}.

    \subsection{Key Features and Requirements}
    Wearable medical devices must meet several essential requirements to distinguish themselves from traditional medical equipment. They must be portable, compact, lightweight, and energy-efficient, enabling prolonged use without frequent recharging. The devices should also be durable, reliable, and biocompatible to withstand the conditions of everyday use. Given the constant interaction between the device and the user, a simple and intuitive user interface is vital~\cite{Lu2020}.

    In general, WMDs share five core characteristics: (1) wireless connectivity for easy data transfer, (2) interactive capabilities that promote intelligent responses to health data, (3) sustainability and robustness under normal wear and tear, (4) simplicity in operation and design for user convenience, and (5) wearability, meaning they should be comfortable to wear for extended periods~\cite{Lu2020}.

\section{Historical Evolution of Wearable Medical Devices}
    \subsection{First-Generation Wearables}

    The first wearable medical devices, often referred to as first-generation wearables, emerged in the 1960s. One of the first examples is the Holter monitor, which was developed to continuously record cardiac activity in a patient for a 24-hour period while they went about their daily routines~\cite{Ates2022}. These early devices primarily focused on monitoring vital signs such as heart rate, blood pressure, and body temperature, and were often designed as portable units that could be worn as watches, shoes, or headsets~\cite{Fotiadis2006}. In the 1990s, the development of wireless wearable medical devices gained traction, enabling continuous monitoring in various applications, including the tracking of NASA astronauts and U.S. Army soldiers~\cite{Luo2024}.

    First-generation wearables were largely limited in their capabilities, with a primary focus on providing basic monitoring. Despite these limitations, they laid the groundwork for future innovations by introducing the concept of continuous health monitoring in real-world environments.

    \subsection{Second-Generation Wearables}
    
    The evolution of wearable medical devices saw a significant leap in the 2010s with the introduction of second-generation wearables. These devices were characterized by their improved flexibility, comfort, and integration with biological fluids for monitoring purposes~\cite{Ates2022}. Second-generation wearables expanded beyond traditional physiological monitoring to include biochemical sensing, such as measuring biomarkers in sweat, saliva, or tears, which provided insight into glucose levels, lactate concentration, and pH levels~\cite{Luo2024}.

    In contrast to their predecessors, second-generation devices often took forms such as on-skin patches, electronic tattoos, tooth-mounted sensors or contact lenses, which made them more comfortable and less intrusive~\cite{Ates2022}. Commercial products like the FreeStyle Libre glucose monitoring system by Abbott and the Cx Sweat Patch by Epicore Biosystems became widely available, demonstrating the potential for real-time biochemical monitoring.

    \subsection{Impact of Technological Advancements}

    Technological advances in materials sciences, sensor miniaturization, and wireless communication have been instrumental in the evolution of wearable medical devices. The integration of flexible electronics, advanced sensors, and energy harvesting technologies has allowed for more sophisticated and user-friendly wearables. The transition from bulky devices to compact, comfortable and efficient devices has significantly improved user adoption and broadened the scope of wearable medical applications~\cite{Ates2022}.

    The evolution of wearables also reflects a shift from simply monitoring vital signs to providing personalized health insights and even predictive analytics. This transformation has been supported by the development of more advanced data processing capabilities, often leveraging cloud computing and artificial intelligence, to provide meaningful health feedback to users and healthcare providers.

\section{Architecture of Wearable Medical Devices}

    \subsection{Core Components of Wearable Medical Devices}
    The architecture of WMDs typically consists of three core components: the wearable sensor module, the data transmission unit, and the data processing and storage system. Together, these components enable continuous health monitoring and effective data analysis, ensuring seamless integration into healthcare workflows~\cite{Ates2022}.
    
        \subsubsection{Wearable Sensor Module}
        The wearable sensor module is responsible for collecting physiological data such as heart rate, body temperature, or blood glucose levels. These sensors are often integrated into comfortable materials that can be worn on the skin, such as patches or textile fabrics. Sensors are designed to be highly sensitive, energy-efficient and non-intrusive to ensure continuous monitoring without discomfort~\cite{Saifuzzaman2021}.
        
        \subsubsection{Data Transmission Unit}
        The data transmission unit handles the communication between the wearable sensor module and external devices, such as smartphones or cloud servers. Typically, data are transmitted wirelessly via technologies such as Bluetooth, Wi-Fi, or NFC~\cite{Guk2019}. This wireless connection allows the data collected by the wearable device to be transmitted in real time to a local device or remote healthcare provider for analysis and monitoring~\cite{Nahavandi2022}.

        \subsubsection{Data Processing and Storage System}
        The data processing and storage system includes both local and cloud-based components. Initially, raw data are processed by a microcontroller embedded in the wearable device itself to extract useful information, such as calculating the heart rate from an electrocardiogram (ECG) signal. Subsequently, the processed data is transmitted to a smartphone or cloud server for long-term storage, advanced analysis, and sharing with healthcare professionals~\cite{Veeravalli2017}.

    \subsection{Power Supply and Energy Harvesting}
    Power supply is a critical aspect of wearable medical devices, as continuous monitoring requires a reliable source of energy. Most wearable devices are powered by rechargeable batteries, but recent advances have explored energy-harvesting techniques to extend battery life or even eliminate the need for charging. Energy can be harvested from various sources, such as body heat, movement, or ambient light, providing a more sustainable power solution for wearable devices~\cite{Ates2022}.

    \subsection{Data Flow and Communication Architecture}
    The communication architecture of wearable medical devices involves multiple layers to ensure efficient data flow and secure information sharing. The data flow begins with the sensors collecting physiological signals, which are then processed locally to filter out noise and extract key features. The processed data are then wirelessly transmitted to a smartphone or gateway device, which then forwards the information to a remote server or cloud platform for further analysis and storage~\cite{Saifuzzaman2021}.

    The data collected by wearable devices is often shared with healthcare providers, allowing real-time monitoring and timely intervention when necessary. This architecture facilitates personalized healthcare by providing actionable information to both patients and healthcare professionals, ultimately leading to improved health outcomes~\cite{Guk2019}.

    \subsection{User Interface and Interactivity}
    The user interface of wearable medical devices is an important aspect of their architecture, as it determines how easily users can interact with the device. Modern wearable devices feature user-friendly interfaces, often accessible through mobile applications, that display health metrics in an easy-to-understand format. The interface may include alerts and notifications to prompt users to take specific actions, such as adjusting their activity level or seeking medical attention based on the collected data~\cite{Nahavandi2022}.


\section{Applications of Wearable Medical Devices}
\subsection{Health Monitoring and Chronic Disease Management}

Examples: ECG, blood glucose monitoring, SpO2 monitoring.

\subsection{Sports and Fitness Applications}

Tracking performance, monitoring recovery.

\subsection{Vital Signs Monitored by WMDs}

ECG, BP, respiratory rate, body temperature, and skin perspiration.

\subsection{Special Cases}

Devices for specific conditions, such as diabetes.

\section{Materials Used in Wearable Medical Devices}

\subsection{Key Material Types}

Graphene, liquid metals, hydrogels, elastomers.

\subsection{Advantages of Different Materials}

Biocompatibility, flexibility, energy harvesting potential.

\subsection{Challenges}

Durability, ensuring long-term stability, comfort.

\section{Advantages and Disadvantages of Wearable Medical Devices}

\subsection{Advantages}
Continuous monitoring, remote healthcare capabilities, non-invasive measurements.

\subsection{Disadvantages}
High costs, privacy issues, data accuracy concerns, battery limitations.

\section{Regulatory Standards and Requirements}
\subsection{FDA and EU Regulations}
Classification of wearable medical devices.

Regulatory requirements for safety, efficacy, and clinical data.
\subsection{ISO and IEC Standards}
Quality management (ISO 13485), risk management (ISO 14971), biocompatibility (ISO 10993).

\subsection{Differences in Regulations Between Medical and Consumer Wearables}
Classification based on the intended use (medical vs. wellness).

\section{Challenges and Future Directions}
\subsection{Technological Barriers}
Sensor limitations, battery life, compatibility.

\subsection{Privacy and Security Concerns}
Managing sensitive patient data.

\subsection{Potential Solutions and Innovations}
AI integration, more durable materials, improved regulatory framework.




\section{Conclusion}
- Recap of the importance of wearable medical devices in healthcare.

- Summary of the key challenges and potential solutions.

- Future prospects for WMDs in revolutionizing personalized healthcare.








% references section

% can use a bibliography generated by BibTeX as a .bbl file
% BibTeX documentation can be easily obtained at:
% http://mirror.ctan.org/biblio/bibtex/contrib/doc/
% The IEEEtran BibTeX style support page is at:
% http://www.michaelshell.org/tex/ieeetran/bibtex/
\bibliographystyle{IEEEtran}
% argument is your BibTeX string definitions and bibliography database(s)
\bibliography{Ref1}
%





% that's all folks
\end{document}


